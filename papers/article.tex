\documentclass{article}
\usepackage{amsmath, amssymb, graphicx, parskip}

\title{A Novel Approach to Forest-Level Splitting for Regression Models}
\author{}
\date{}

\begin{document}

\maketitle

\begin{abstract}
\end{abstract}

\section{Introduction}
The remainder of this paper is organized as follows: Section~\ref{sec:relatedwork} reviews related work, Section~\ref{sec:methodology} introduces Dynaforest’s methodology, Section~\ref{sec:experiments} describes our experimental setup, and Section~\ref{sec:results} presents results and analysis. Finally, Section~\ref{sec:conclusion} concludes with insights and future directions.

\section{Related Work}\label{sec:relatedwork}

\section{Dynaforest Methodology}\label{sec:methodology}

\subsection{Overview}

\subsection{Mathematical Foundation}
In regression tree analysis, determining the optimal split point within a 
leaf node involves selecting a value, denoted as $\alpha$, that minimizes
the sum of loss functions over all observations in the node. This is 
mathematically represented as minimizing 
\[
\sum_{i=1}^n L(y_i, \alpha),
\]
where $L$ denotes the loss function and $y_i$ are the observed values. For 
regression tasks employing the mean squared error (MSE) as the loss function, $\alpha = \bar{y}$.

Building upon this, it has been demonstrated [BY WHOM] that the objective function can be 
simplified to:
\[
\sum_{i=1}^n y_i^2 - n\bar{y}_n^2 = \sum_{i=1}^n y_i^2 - \frac{(\sum_{i=1}^n y_i)^2}{n}.
\]

This formulation is computationally advantageous, as it allows for the 
calculation of optimal sums in $O(n)$ time after sorting, thereby enhancing efficiency. I
n our approach, we extend this methodology by incorporating predictions from an ensemble 
of M trees within a specified window. The computation is performed as follows:

\begin{align*}
    E &= \\
    \sum_{i=1}^n (y_i - \frac{M o_i + \frac1n \sum_{i=1}^n y_i}{M + 1})^2 &= \\
    \sum_{i=1}^n (y_i - \frac{M o_i + \bar{y}_n}{M + 1})^2 &= \\
    \sum_{i=1}^{n} y_i^2 - \frac{2M y_i o_i + 2y_i \bar{y}_n}{M + 1} + \frac{(M o_i + \bar{y}_n)^2}{(M + 1)^2} &= \\
    \sum_{i=1}^{n} y_i^2 - \frac{2M}{M + 1} \sum_{i=1}^{n} y_i o_i - \frac{2}{M + 1} n\bar{y}_n^2 + \frac{M^2}{(M + 1)^2} \sum_{i=1}^{n} o_i^2 + \frac{2M}{(M + 1)^2} \bar{y}_n \sum_{i=1}^{n} o_i + \frac{\sum_{i=1}^{n} u_n^2}{(M + 1)^2} &= \\
    \sum_{i=1}^{n} y_i^2 - \frac{2M}{M + 1} \sum_{i=1}^{n} y_i o_i - \frac{2}{M + 1} n\bar{y}_n^2 + \frac{M^2}{(M + 1)^2} \sum_{i=1}^{n} o_i^2 + \frac{2M}{(M + 1)^2} \bar{y}_n \sum_{i=1}^{n} o_i + \frac{(\sum_{i=1}^{n} y_i)^2}{n(M + 1)^2} &= \\
    \sum_{i=1}^{n} y_i^2 - \frac{2M}{M + 1} \sum_{i=1}^{n} y_i o_i - \frac{2(\sum_{i=1}^{n} y_i)^2}{n(M + 1)} + \frac{M^2}{(M + 1)^2} \sum_{i=1}^{n} o_i^2 + \frac{2M \sum_{i=1}^n y_i}{n(M + 1)^2} \sum_{i=1}^{n} o_i + \frac{(\sum_{i=1}^{n} y_i)^2}{n(M + 1)^2}
\end{align*}

Again, we only need to pre-calculate cumulative sums for \[
    \sum_{i=1}^{n} y_i, 
    \sum_{i=1}^{n} o_i, 
    \sum_{i=1}^{n} o_i y_i,
    \sum_{i=1}^{n} y_i^2,
    \sum_{i=1}^{n} o_i^2 \]

[Add a few sentences here, AND mention the possibility of replacing the mean with some specified value]

\subsection{Algorithm}

\section{Experimental Setup} \label{sec:experiments}

\subsection{Datasets}

\subsection{Metrics}

\subsection{Baselines}

\section{Results and Discussion}\label{sec:results}
\subsection{Prediction Accuracy}

\subsection{Interpretability}

\subsection{Efficiency}

\section{Conclusion}\label{sec:conclusion}

\bibliographystyle{plain}

\end{document}
