\documentclass{article}
\usepackage{amsmath}
\usepackage{amssymb}
\usepackage{parskip}

\begin{document}

\title{Loss Math}
\author{}
\date{}

\maketitle

\section*{Algorithm}
We let $w$ be the weight of a given parent leaf, and $w_L$, $w_R$ be the weights of the left and right children of the parent leaf. 
For ease, denote $\Delta_L = w - w_L$ and $\Delta_R = w - w_R$. Our prediction $\hat{y}^{(t - 1)} = o_i + \eta w$. 
We increment our left leaf by some value $\Delta_L$ to get leaf value $w_L$. Our updated prediction is thus
$\hat{y}^{(t)} = o_i + \eta w + \eta \Delta_L$. We will actually set $\hat{y}^{(t)} = o_i + \eta w + \epsilon \eta \Delta_L$, 
but that is not needed for this calculation.

By Taylor Expansion, we have that \begin{align*}
    l(y_i, \hat{y_i}^{(t)}) &= \\
    l(y_i, \hat{y_i}^{(t - 1)} + \eta \Delta_L) &= \\
    l(y_i, \hat{y_i}^{(t - 1)}) + \eta \Delta_L \frac{\partial l(y_i, \hat{y_i}^{(t - 1)})}{\partial \hat{y_i}^{(t - 1)}} + \frac{\eta^2 \Delta_L^2}{2} \frac{\partial^2 l(y_i, \hat{y_i}^{(t - 1)})}{\partial (\hat{y_i}^{(t - 1)})^2}&= \\
    l(y_i, \hat{y_i}^{(t - 1)}) + \eta \Delta_L g_i + \frac{\eta^2 \Delta_L^2}{2} h_i
\end{align*}

We also add a regularization term of $\frac{1}{2}\lambda(w + \Delta_L)^2 = \frac{1}{2} \lambda (w^2 + 2w\Delta_L + \Delta_L^2)$
Combining terms and removing constants with respect to $\Delta_L$, we wish to minimize \[
 \eta \Delta_L g_i + \frac{\eta^2 \Delta_L^2}{2} h_i + \frac{1}{2} \lambda (2w\Delta_L + \Delta_L^2)
\]

Taking the derivative with respect to $\Delta_L$ and adding in all values on the left side, we have that \begin{align*}
    \eta G_L + \eta^2 \Delta_L H_L + \lambda w + \lambda \Delta_L = 0 &\implies \\
    \Delta_L \left(\eta^2 H_L + \lambda \right) = -\eta G_L - \lambda w &\implies \\
    \Delta_L = -\frac{\eta G_L + \lambda w}{\eta^2 H_L + \lambda}
\end{align*}

We know that for a quadratic function in the form of $\frac{1}{2}bx^2 + ax$, the minimum value is $- \frac{a^2}{2b}$. 
We have that $b = \eta^2 H_L + \lambda$, and $a = \eta G_L + w\lambda$. So, the gain at the leaf is \[
\frac{1}{2} \frac{(\eta G_L + w\lambda)^2}{\eta^2 H_L + \lambda}
\]

The determination, then, is whether to begin with w or the mean of y as the initial value.

\section*{Old algorithm}

For a given leaf and split, we have that minimum loss is given by \[
\min_{\gamma} \sum_{i=1}^{n} \mathcal{L}(y_i, \frac{\alpha o_i + \gamma}{\alpha + 1})
\]

We can set the "plain optimum" to be \[
\gamma^* = \arg \min_\gamma \sum_{i \in \text{leaf}} \mathcal{L}(y_i, \gamma)
\]

And the "ensemble optimum" to be \[
\tilde{\gamma}^* = \arg \min_\gamma \sum_{i \in \text{leaf}} \mathcal{L}(y_i, \frac{\alpha o_i + \gamma}{\alpha + 1})
\]

Letting a star denote the optimal value that combines ensemble predictions, I split at $A^*$ and define the 
the left leaf value as 
\[
B_{\text{leaf}} = (1 - \delta)(\gamma^*) + \delta \tilde{\gamma}^*
\]
With a split at $A^*$

The split is defined at:
\[
\arg\min_A \sum_{i=1}^n \mathcal{L}(y_i, \frac{\alpha o_i + (1 - \delta)\sum_{i=1}^n \arg\min_{\gamma} \mathcal{L}(y_i, \gamma) + \delta \sum_{i=1}^n \arg\min_\gamma \mathcal{L}(y_i, \frac{\alpha o_i + \gamma}{\alpha + 1})}{\alpha + 1})
\]

Where we define \[
    \sum_{i=1}^n (1 - \delta)\arg\min_{\gamma} \mathcal{L}(y_i, \gamma) + \sum_{i=1}^n \delta \arg\min_\gamma \mathcal{L}(y_i, \frac{\alpha o_i + \gamma}{\alpha + 1}) = B_n
\]

\section*{Generic Algorithm}
Begin with initial value \[
    \gamma_0 = \arg\min_{\gamma} \sum_{i=1}^n \mathcal{L}(y_i, \gamma)
\]

Find error at split A \[
\gamma^* = \gamma_0 - \eta \frac{G}{H + \lambda}
\]

Given \[
\mathcal{L}(y_i, \frac{\alpha o_i + \gamma_0}{\alpha + 1})
\]

We have that $g_i$ = \[
\frac{1}{\alpha + 1} \frac{\partial \mathcal{L}(y_i, z)}{\partial z}\big|_{z = \frac{\alpha o_i + \gamma_0}{\alpha + 1}}
\]

And $h_i$ = \[
\frac{1}{(\alpha + 1)^2} \frac{\partial^2 \mathcal{L}(y_i, z)}{\partial z^2}\big|_{z = \frac{\alpha o_i + \gamma_0}{\alpha + 1}}
\]

So that $G = \sum_{i \in \text{leaf}} g_i$ and $H = \sum_{i \in \text{leaf}} h_i$ 

So, we have that \begin{align*}
    \frac{G}{H} &= \\
    \frac{\sum_{i \in \text{leaf}} g_i}{\sum_{i \in \text{leaf}} h_i} &= \\
    \frac{\sum_{i \in \text{leaf}} \frac{1}{\alpha + 1} \frac{\partial \mathcal{L}(y_i, z)}{\partial z}\big|_{z = \frac{\alpha o_i + \gamma_0}{\alpha + 1}}}{\sum_{i \in \text{leaf}} \frac{1}{(\alpha + 1)^2} \frac{\partial^2 \mathcal{L}(y_i, z)}{\partial z^2}\big|_{z = \frac{\alpha o_i + \gamma_0}{\alpha + 1}}} &= \\
    (\alpha + 1)\frac{\sum_{i \in \text{leaf}} \frac{\partial \mathcal{L}(y_i, z)}{\partial z}}{\sum_{i \in \text{leaf}} \frac{\partial^2 \mathcal{L}(y_i, z)}{\partial z^2}}
\end{align*}


\section*{MSE}
The gradient we have as \[
    \frac{\partial \mathcal{L}(y_i, z)}{\partial z}\big|_{z = \frac{\alpha o_i + \gamma_0}{\alpha + 1}} = -2 \left( y_i - \frac{\alpha o_i + \gamma_0}{\alpha + 1} \right)
\]

And hessian we have as \[
    \frac{\partial^2 \mathcal{L}(y_i, z)}{\partial z^2}\big|_{z = \frac{\alpha o_i + \gamma_0}{\alpha + 1}} = 2
\]

So that \begin{align*}
    -\frac{G}{H} &= \\
    \sum_{i=1}^n (\alpha + 1)y_i - \alpha o_i - \gamma_0 &= \\
    \sum_{i=1}^n (\alpha + 1)y_i - \sum_{i=1}^n \alpha o_i - \sum_{i=1}^n \sum_{i=1}^n \frac{y_i}{n} &= \\
    \alpha \sum_{i=1}^n (y_i - o_i)
\end{align*}

\section*{New version of loss-agnostic algorithm}
The algorithm can be desribed as following:
\[
\hat{y}_i = \phi(x_i) = \sum_{k=1}^K f_k(x_i) = \sum_{k=1}^K w^k_{q_k(x_i)}
\]

We define the loss as \[
\mathcal{L}(\phi) = \sum_{i=1}^n \mathcal{L}(y_i, \phi(x_i)) + \sum_{i=1}^k \sum_{j=1}^{T_i} 1 + \frac{1}{2} \lambda (w^k_j)^2
\]

Where $T_i$ represents the number of leaves in the $i$th tree and $w^k_j$ represents the value of the $j$th leaf in the $k$th tree.

We begin by creating all trees to the warmup depth, and then update each one at a time. Suppose
we begin with tree m WLOG, updating at leaf $j$. Denote $I_j = \{i: q_k(x_i) = j\}$. Similarly, denote $I_{j, l}$ to be the points
split to the left of leaf j in tree k after iteration t - 1.

Then, 

ONLY WORK WITH ONE SPLIT LEAF AT A TIME, I THINK?
\begin{align*}
    \mathcal{L}^{(t)} &=
    \begin{split}
        \left(\sum_{i \notin I_j} l(y_i, \hat{y}_i^{(t - 1)}) + \sum_{k \neq m} \Omega(f_k) \right) + 1 + \frac{1}{2} \lambda w_{j, l}^2 + \frac{1}{2} \lambda w_{j, r}^2 - \frac{1}{2} \lambda w_{j}^2 +  \\
    \sum_{i \in I_{j,l}} l\left(y_i, w_{j, l} + \sum_{k \neq m} f_k(x_i)\right) +  \sum_{i \in I_{j,r}} l\left(y_i, w_{j, r} + \sum_{k \neq m} f_k(x_i)\right)
    \end{split} &= \\
      \frac{1}{2}
\end{align*}

\end{document}